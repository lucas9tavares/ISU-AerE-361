\documentclass{article}


% To remove paragraph indentation
\usepackage{parskip}
% Adds the url command
\usepackage{hyperref}

% Update these:
\title{Lab 7 Report}
\date{03-04-2024}
\author{Lucas Tavares}


\begin{document}

\maketitle


\newpage

\section{Exercise 1 - Midpoint Rule}

For the Midpoint Rule function, no loops were used, just a single line of simple algebra. Independent of the input values for a and b, and disregarding the complexity of the \texttt{integrand} function, \texttt{midpoint} runs in constant time. Thus, its complexity is \textit{O(1)}.

\section{Exercise 2 - Simpson's 1/3}

For the Simpson's 1/3 function, again, no loops were used, just a single line of simple algebra. Independent of the input values for a and b, or the extra elements in the equation compared to the Midpoint Rule equation, and disregarding the complexity of the \texttt{integrand} function, \texttt{simpson\_13} runs in constant time. Thus, its complexity is \textit{O(1)}.

\section{Exercise 3 - Simpson's 3/8}

Lastly among the ''simple'' numerical integration methods, for hte Simpson's 3/8 function, no loops were used, just a single line of simple algebra. Independent of the input values for a and b, or the extra elements in the equation compared to the Simpson's 1/3 equation, and disregarding the complexity of the \texttt{integrand} function, \texttt{simpson\_38} runs in constant time. Thus, its complexity is \textit{O(1)}.

\section{Exercise 4 - Gauss Quad}

Considering the Gauss Quad function includes a single \texttt{for} loop, its performance will grow linearly and in direct proportion to the size of the inputs. Still, as the bounds of the integration increase, there will not be an increase in the number of code lines executed. The only term that can change things is n. The loop will always iterate n times, so if n increases by 1, two extra lines of code will be executed. Thus, the complexity of the \texttt{gauss\_quad} functions equal \textit{O(n)}.


\newpage

\section{Sources and Resources}

\begin{flushleft}
\begin{itemize}
    \item{\texttt{Gauss\_Quad\_array.txt}: Provided in the lab 7 repo}
    \item{\url{https://ctan.mirrors.hoobly.com/info/latexcheat/latexcheat/latexsheet.pdf}: Provided in the lab manual}
\end{itemize}
\end{flushleft}


\end{document}
